\chapter{Preliminary test}\label{ch:PreTest}
This test was conducted to investigate the state of the art regarding AirConsole games. Additionally, the test experimented with eye tracking to see if it was viable for the final experiment of the project.

\section{Test setup}
Players were seated in such a way that they would be facing a PC screen placed on a table. A camera above and behind the PC screen recorded the players. Each test participant was given a smartphone to use with the AirConsole platform. Test participants played a designated game uninterrupted to emulate a casual playing environment. Screen recording software was installed on the participant’s smartphone as well as the PC screen. Eye tracking was also used to track the eye focus of the participant, whose smartphone was also recorded.

\subsection{Materials used}
\begin{itemize}
\item Three smartphones
\item Tobii eyeX eye tracker
\item Video camera
\item Laptop
\item Screen recording software on phone
\item Screen recording software on the laptop that includes eye tracking
\end{itemize}
	
\section{Games used}
We investigated already existing games on AirConsole to see how many times people looked between the main screen and the phone. In the test we used the two games:
\begin{itemize}
\item Mockingbird
\item Tic Tac Boom
\end{itemize}

Mockingbird is a game in which the player controls a bird by tapping the screen in order to make the bird fly upwards. By tilting the phone, the bird’s flight changes from going directly upwards to sideways parabola paths. The controls are shown on the main screen during startup of the game, and they stay on the screen of the phone throughout gameplay, as shown in Figure \ref{fig:MBird2}. The objective of the game is to collect eggs while avoiding other birds.

\begin{figure}
\centering
\includegraphics[width=\textwidth]{figures/birdSelect}
\caption{Mockingbird start up screen}\label{fig:MBird1}
\end{figure}

\begin{figure}
\centering
\includegraphics[width=\textwidth]{figures/birdPlay} 
\caption{Mockingbird's controls instruction}\label{fig:MBird2}
\end{figure}

Tic Tac Boom is a game in which a player controls a character in an obstacle-filled arena. The character is controlled with buttons on the smartphone. There are controls for four-directional movement, placement of bombs, and three power-ups. The power-ups light up when they are available and are greyed out when unavailable. The controller layout is seen in Figures \ref{fig:TTB1} and \ref{fig:TTB2}. The objective of the game is to blow up the opponent with bombs, while avoiding getting blown up. Certain obstacles in the arena can be destroyed by bombs, which is required in order to get to the opponent.

\begin{figure}
\centering
\includegraphics[width=\textwidth]{figures/TTBcontrol}
\caption{Tic Tac Boom's controller with no available power-ups}\label{fig:TTB1}
\end{figure}

\begin{figure}
\centering
\includegraphics[width=\textwidth]{figures/TTBpowerUp}
\caption{Tic Tac Boom's controller with an available power-up}\label{fig:TTB2}
\end{figure}

Figures \ref{fig:birdTable} and \ref{fig:TTBtable} show how many times the player looked away from screen. 

\begin{figure}
\centering
\includegraphics[width=\textwidth]{figures/birdTable} 
\caption{A table of the amount of times the participants looked at the smartphone while playing Mockingbird}\label{fig:birdTable}
\end{figure}

\begin{figure}
\centering
\includegraphics[width=\textwidth]{figures/TTBtable} 
\caption{A table of the amount of times the participants looked at the smartphone while playing Tic Tac Boom}\label{fig:TTBtable}
\end{figure}

When comparing the numbers in the two tables, it can be seen that participants looked at their phones considerably less when playing Mockingbird compared to Tic Tac Boom. This indicated that simpler controls would lead players to focus on the main screen of the game. Inversely, by having more controls the players looked down on their screens more often to verify that the correct area of the controller was tapped. 
