\chapter{Premiminary test}\label{ch:PreTest}
\section{Test setup}
Players sit facing a PC screen on a table. A camera above and behind the PC screen records the players. Each test participant is given a smartphone to use with the AirConsole platform. Test participants play a designated game uninterrupted to emulate a casual playing environment. Screen recording software is installed on one participant’s smartphone as well as the PC screen. Eye tracking is also used to track the eye focus of the participant, whose smartphone is also recorded.


\subsection{Materials used}
\begin{itemize}
\item Smartphones
\item Tobii eyex controller
\item Camera
\item Laptop
\item Recording software for both phone and laptop
\begin{itemize}
	\item Smartphone recording software: AZ screen recorder
	\item Recording software on the laptop that includes eye tracking
\end{itemize}
\end{itemize}

	
\section{Games used}
We investigated already existing games on AirConsole to see how many times people looked between the main screen and the phone. In the test we used the two games:
\begin{itemize}
\item Mockingbird
\item Tic Tac Boom
\end{itemize}

Mockingbird is a game in which the player controls a bird. You do so by tapping the screen in order to make the bird fly upwards. By tilting the phone, the bird’s flight changes from directly flying upwards and falling to sideways parabola paths. The controls are shown on the main screen during startup of the game, and they persist on the screen of the phone throughout gameplay, as shown in the figures below. The objective is to collect eggs while avoiding other birds.
\begin{figure}
\centering
\includegraphics[width=\textwidth]{figures/birdSelect} \label{fig:MBird1}
\caption{Mockingbird start up screen}
\end{figure}
\begin{figure}
\centering
\includegraphics[width=\textwidth]{figures/birdPlay} \label{fig:MBird2}
\caption{Mockingbird's controls instruction}
\end{figure}

Tic Tac Boom is a game where you control a character in an obstacle-filled arena. You control the character with buttons on the smartphone, having controls for 4-directional movement, placement of bombs, and 3 individually activatable powerups. The powerups light up when they are available and are greyed out when unavailable. The controls layout is seen in the figures below, with the leftmost powerup being unavailable and available respectively. The objective of the game is to blow up your opponent with bombs, while avoiding getting blown up by your opponent and yourself. The arena can be destroyed by bombs, which is required in order to get to your opponent.

\begin{figure}
\centering
\includegraphics[width=\textwidth]{figures/TTBcontrol} \label{fig:TTB1}
\caption{Tic Tac Boom's controller}
\end{figure}

\begin{figure}
\centering
\includegraphics[width=\textwidth]{figures/TTBpowerUp} \label{fig:TTB2}
\caption{Tic Tac Boom's controller with available power up}
\end{figure}

The table shows how many times the player looked off the screen and looked else where. 

\begin{figure}
\centering
\includegraphics[width=\textwidth]{figures/birdTable} \label{fig:birdTable}
\caption{A table of the amount of times the participants looked at the smartphone while playing Mockingbird}
\end{figure}

\begin{figure}
\centering
\includegraphics[width=\textwidth]{figures/TTBtable} \label{fig:TTBtable}
\caption{A table of the amount of times the participants looked at the smartphone while playing Tic Tac Boom}
\end{figure}

When comparing the numbers in the two tables, we can see that participants looked at their phone considerably less when playing Mockingbird contra Tic Tac Boom. This indicates that simpler controls will lead players to focus on the main screen of the game. Inversely, by having more controls to manipulate, including powerups and special abilities, the player looks more down on their screen to verify they are pressing the correct area of the controller. 
