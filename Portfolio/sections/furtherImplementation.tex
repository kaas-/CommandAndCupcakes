\chapter{Further implementation}\label{ch:further_implementation}
A game was developed to test how different attention cues work on directing the attention of players. This means that with the results gathered in Chapter \ref{ch:EvalResults}, further iterations of the game can be made.

Many features were considered and ultimately cut from the version used in the experiment. This was primarily due to time constraints, but it was also because they did not contribute to the experiment, but instead would be implemented to improve the game.

Since AirConsole is a video game platform, the game made in this project would have to be further developed into a higher quality product if it was to be published on the platform later. Therefore, this chapter describes features not implemented during development, plus further features found throughout the experiment.

\section{Digging for the treasure}\label{sec:digging}
The version of the game used in the test has the win condition of gathering all nine map pieces. 

The original win condition was to gather all nine map pieces, and then dig up the treasure in a square the map indicated. To implement this win condition, further features would have to be in the game:
\begin{itemize}
\item Functionality to determine where the treasure is buried
\item A treasure map, split into pieces
\item A menu on the controller to view collected treasure map pieces
\item An action used to dig
\end{itemize}

Currently, the map pieces are randomly hidden inside objects on the island. Likewise, the treasure was also conceptualised to be hidden in a random location on the island. To find its exact location, a player would need to gather all the map pieces. Each collected map piece would illustrate a part of the island, and when all nine map pieces were collected, then it would be possible to easily discern which part of the island the map illustrates. Additionally, an X would be placed on the now complete map, to indicate the exact tile where the treasure would be hidden.

In order for the player to review which map pieces were in their possession, a separate screen would be implemented. This screen would have a 3x3 grid, with each cell being able to contain a collected map piece. Each cell would be unique to a specific map piece, thus when a map piece is collected it would automatically be shown in the correct cell.

There would be a button to switch between the planning screen and the map screen, as described in Chapter \ref{ch:visualDesign}.

When the location of the treasure was known by the player, they would have the ability to dig it up. This would be implemented as an additional action available to all players. The icon for the action button would have looked like Figure \ref{fig:dig}.

\begin{figure}[h!]
	\centering
	\includegraphics[width=0.2\textwidth]{figures/dig.png}
	\caption{Dig action icon \label{fig:dig}}
\end{figure}

The dig button was a unique action in that it would take two actions to execute instead of the regular one. This meant that the player would use their whole turn to start digging in the tile they were currently located in. However, all other players would see that a player had started to dig, and have the chance to stop them by attacking. If the player was left alone until the start of their next turn, then they would have successfully dug up the treasure and won the game. 

The dig action did not require a player to have all the map pieces in order to start digging somewhere. If a player has an idea of which square the treasure could be located in from a partially completed map, then they could still dig to try their luck. If a player guessed correctly and was not interrupted in digging, they would win the game, since the treasure would be hidden at the start of the game.

This win condition would increase the game length, as it is adding more tasks to do for the individual player on top of what they are already supposed to do. This means that to keep games short, fewer map pieces should be collected to get the final map. This way, the game could be played across multiple matches to find a final winner. Playtesting would be needed in order to find the proper balance between the number of map pieces and the match length.

\section{Game changes from results}\label{sec:changes_from_results}
The game's current state of implementation is a tool to be used to investigate the problem statement. However, further development outside of the project period could be done to change the game from a testing tool to a piece of entertainment media. 

The current turn order is randomly determined at the start of each round. This design decision was directly tied to the experiment so people had to pay attention at all times to the game if players did not want to miss something.

During the experiment, some participants commented negatively on the random turn order, stating that it was confusing. One way to solve this would be to have a fixed turn order, so that players would have a reliable sense of when it was their turn. This would also prevent some players having very long wait times between their turns. For example, a player starting in one round and being the last player in the following round would have to wait through six other turns. The longest wait time recorded during the experiment was approximately 90 seconds, with the shortest being approximately five seconds. Another way would be to display the turn order on the main screen to keep the randomness and unpredictability as a feature, but not hiding the information from the players. With this information in mind, players could more effectively plan ahead.