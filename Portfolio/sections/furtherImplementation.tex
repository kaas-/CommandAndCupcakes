\chapter{Further Implementation}\label{ch:further_implementation}
The project made a game to test how different attention cues work on directing the attention of players. This means that with the results gathered in the previous chapter, an iteration of the game can be made.

However, many features were considered and ultimately cut from the version used in the experiment. This was primarily due to time constraints, but it was also because they did not contribute to the experiment, but instead would be implemented to make the game of higher quality.

Since AirConsole is a video game platform, the game made in this project would have to be further developed into a higher quality product if it were to be published on the platform later. Therefore, this chapter describes features cut during development to save time, plus further features found through the experiment.

\section{Digging for the Treasure}
The version of the game used in the test has the win condition of gathering all nine map pieces. This was to actually have a goal for the players to strive for during testing.
However, the original win condition was to gather all nine map pieces, and then dig up the treasure in the square the map indicated. To implement this win condition, further features would have to be in the game:
\begin{itemize}
\item Functionality to determine where the treasure is buried
\item A treasure map, split into pieces
\item A menu on the controller to view collected treasure map pieces
\item An action used to dig in a square
\end{itemize}

Right now, the map pieces are randomly hidden inside objects on the island. Likewise, the treasure was also conceptualised to be hidden in a random location on the island. To find its exact location, a player would need to gather all the pieces of the map. 