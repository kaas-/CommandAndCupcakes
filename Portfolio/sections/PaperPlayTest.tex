\chapter{Paper prototype tests}\label{ch:paperPlaytest}
Before Treasure Hunt was developed, several paper prototype tests were conducted. The first paper test was an external test using ten participants, and the second was an internal paper playtest which sought to investigate whether the planned game mechanics worked in practice.

\section{External grid and interaction test}
In order to test how to best place interactive objects on the grid, as well as whether users prefer to tap actions on their smartphone screen or drag them to a designated place, a paper test was conducted. There were ten participants in this test, which consisted of two parts: one which sought to investigate the best way to represent an interactive object on a grid, and one which sought to investigate whether users preferred to tap or drag actions on the smartphone. For both parts of this test, three people were present: a test participant; a test conductor, who explained the test procedure to the participants, showed them the relevant paper prototypes and asked questions; and a note taker, who was responsible for data collection. This test setup can be seen in Figure \ref{fig:papersetup}.

\begin{figure}[h!]
	\centering
	\includegraphics[width=\textwidth / 2]{figures/PaperTestSetup.png}
	\caption{Setup for the paper test. Not pictured is the note taker. \label{fig:papersetup}}
\end{figure}

\subsection{Testing object placement on a grid}
For the first part of the test, twelve images were produced, depicting six different ways to represent a character and an interactive object on a grid. For each of these six methods, two versions were used: one in which the character was placed on the same space as the object (and could thus interact with it), and one in which they were on a different space. The participants were asked the same two questions for each of these twelve images:

\begin{enumerate}
    \item From where the character is currently standing, can they interact with the object?
    \item If the character is on the same space as the object, can they move away from it---or, if they are in a space adjacent to the object, can they move onto it?
\end{enumerate}

The images were numbered from 1 to 12, and depicted different ways to represent an object and a character on a grid. The even numbers all depicted a character which was on a different space than the object, and the uneven numbers all depicted a character which was not. These images can all be seen in Figure \ref{fig:papergrid}.

\begin{figure}[h!]
	\centering
	\includegraphics[width=\textwidth]{figures/paper_0.png}
	\caption{The twelve images which were presented to the test participants. \label{fig:papergrid}}
\end{figure}

Images 1-4 depict a character and an object which are both placed on the squares of a grid rather than on the vertices. In images 1 and 2, the object is centred on its square and takes up a majority of its space, whereas the object is diminished in size and pushed aside in images 3 and 4. In images 5-8, both the character and the object are on the vertices rather than the square. The difference is that the object is centred on the intersection between the vertices in images 5 and 6, and pushed aside in images 7 and 8 so that the intersection is visible. Images 9 and 10 depict a character which is on the vertices and an object which is on the squares, and images 11 and 12 depict a character which is on the squares and an object which is on the vertices. In the last four versions, the object is interactive when the character is placed on any of the four adjacent spaces, whereas the other eight versions only provide one available interaction space for an object.

\subsection{Testing whether users prefer to tap or drag objects}
The second part of the test investigated how users interacted with the interface. Participants were presented with a paper mock-up of the interface, seen in Figure \ref{fig:mockup}. They were then given four tasks, and it was noted whether the participant tapped the available actions or dragged them to one of the two circles.

\begin{figure}[h!]
	\centering
	\includegraphics[scale=0.7]{figures/paper_1.jpg}
	\caption{Paper mock-up representing a computer screen (above) and a smartphone screen (below). Note that the elements with a thick border were movable. \label{fig:mockup}}
\end{figure}

\subsection{Results}
After the data was collected for the first portion of the test, it was counted for each of the twelve images how many times a participant failed to see that they could interact with the object, and how many times they failed to see that they could move. This gave a clear indication of which of the different proposals for object and character placement were intuitive to the majority of the participants.

The test revealed that most of the participants made wrong assumptions about how the character should move and interact with the objects when shown images 1, 2, 5 and 6. In images 5 and 6, an object intersected with the walking path of the character, which may be the reason why participants judged that they could not move past the object. In images 1 and 2, the object occupied the majority of its tile, which served as another source of confusion, since many players judged that they could not move onto it. They also though that they could interact with it while being on an adjacent tile. All of the aforementioned issues occurred when a participant was shown the first image in a sequence (images 1 and 5), when the character was not yet on the same tile as the object. Those issues were resolved when the second image (number 2 and 6) was shown. However, this indicates that if users were introduced to a game with a design corresponding to the ones shown in images 1, 2, 5 and 6, they could have wrong ideas on how to move within the game field and interact with the objects. This could result in them not being aware of all their opportunities.

When shown images 3 and 4, in which the object was pushed aside on the square, the participants understood how to move within a grid and how to interact with an object, with only a few exceptions. The results were similar when shown images 7 and 8. However, in this case, once the character was on the right spot, a few participants did not know that they could interact with the object or that they could move past it. 

In images 9-12, when a character was on one of the vertices and the object was on a square, or the other way around, some participants were unclear if they could interact with the object. Specifically in images 10 and 12, some participants thought they were in a wrong spot to do so, when in fact the opposite was the case. There was, however, no confusion regarding to where the character could move.

The results of the second part of the test have shown that the majority of the participants (7 out of 10) preferred tapping an icon when given a specific task, rather than dragging it. However, these results might, in part, be influenced by the paper setup, since in this case it might have been easier to tap an object of interest, compared to dragging it, which might not be the case when transferred to a smartphone. 

It is also clear that the participants understood what the different icons on the phone screen setup meant when asked to accomplish a task in the game. All of the participants also understood that digging took two actions, since the two shovel icons were merged together. 

\section{Internal paper prototype test}
A second prototype test was conducted amongst group members to test whether the intended game mechanics worked in practice.

\subsection{The prototype}
\begin{figure}[h] 
\centering 
\includegraphics[width=0.5\textwidth]{figures/PPtestArena}
\caption{The prototype's arena} \label{fig:PPtest1}
\end{figure}

The board consisted of a 5x5 grid, as can be seen on Figure \ref{fig:PPtest1}. Some tiles had an object inside, which was interactable. Nine of the objects contained a map piece that was received when the player interacted with the object. The objects were represented by white pieces of paper with skulls, trees and other drawings on them. The player characters were represented by pieces of paper with a coloured outline in red, blue, green and yellow colours.

The objective of the game was to find pieces of a treasure map that were hidden in the objects of the map. When a player had all the map pieces, the location of the treasure was revealed. By digging up the treasure, a player would win.


There was a small "x" on the back of nine of the objects. This was to indicate whether there was a map piece hidden in that object.

\begin{figure}[h]
\centering
\includegraphics[width=0.8\textwidth]{figures/PPtestControls}  
\caption{The prototype's player controls}\label{fig:PPtest2}
\end{figure}

Each player had a mock controller which was divided into two parts, see Figure \ref{fig:PPtest2}. The left side displayed collectable map pieces that the player possessed. The right part was the action menu, which had buttons for all the controls. When pressing a button, its icon was displayed in one of the two empty circles in the top, starting from the leftmost one. Each button represented an action that a player could take on their turn. The actions in the top row from left to right were Search, Attack, and Dig. Dig was special because it was two icons bound together, since it took the whole turn to do this action.

When both actions were selected, pressing the checkmark button would confirm the selection, and the turn was executed.

\begin{figure}[h!]
\centering
\includegraphics[width=0.4\textwidth]{figures/PPtestMap} 
\caption{The prototype's treasure map}\label{fig:PPtest3}
\end{figure}

The left part of the screen was for containing the treasure map pieces that the player collected. When collecting one of the nine map pieces hidden on the board, the piece would appear in the correct position and orientation on the grid. By having enough map pieces, the player was able to identify where the treasure was. The treasure map division can be seen in Figure \ref{fig:PPtest3}.

The small cross above the map pieces was used when a player had collected all the map pieces. Then the cross would appear on the map to show the player where the treasure was hidden.

\subsection{The play test}
The game was played for a few rounds with four players. Each player got a turn each round, with the turn order randomly determined at the start of each round.  After a few rounds, the following issues with the prototype were identified.

\subsubsection{Combat}
In the prototype, the act of attacking was done by using one action to move into the square of an opponent, and then another action for attacking. This caused the defender to lose all their map pieces to the attacker and become \textit{stunned}, which meant losing access to one of their actions in their following turn. This was problematic as this could lead to endless cycles of attacking between two players, and also have no consequence for the attacker.

To replace this, a different combat system was designed. The attack action was removed, thus making the combat automatically initiate when the player, who had the current turn, entered the square of an opponent. Additionally, instead of the attacker winning automatically, a mini-game was designed which both attacker and defender would play in order to determine who would win the combat. Multiple numbered dots would appear on both players’ smartphones, and the objective would be to press all the dots in correct sequence before the opponent. The loser would become stunned. Additionally, they would move 2 spaces away from the winning player in a predetermined direction. If the attacker won, they would steal all the defender’s map pieces. If the defender won, they would keep their own map pieces, but would not steal pieces from the attacker.

This design could have resulted in ties, in case both players completed at the exact same time. The likelihood of that could be manipulated by how precisely time would be recorded in the minigame, with more precise timekeeping resulting in fewer cases of ties. However, even with a low probability this situation could still occur. Therefore, in case of a tie, both players would keep their map pieces, but both would also become stunned and would be moved away from each other.

Only parts of this design were implemented due to time constraints. The final version of the minigame is described in the report.

\subsubsection{Turn order}
Having the turn order randomly determined at the start of each round could result in a player getting the last turn in a round, and then having the first turn in the following round. This effectively made the player have two turns in a row, which seemed unfair.

To prevent this, two different solutions were proposed:
\begin{itemize}
\item Make the turn order the same across rounds, starting with player 1, then player 2 and so on.
\item Adding logic into the turn order which made sure that a player could not be first in a round if they had the last turn in the previous round.
\end{itemize}

It was decided to implement the second solution based on the problem formulation.

\subsubsection{Controller}
Having the controls and the collected map pieces on the same screen resulted in having a limited space for both. To solve this, a new controller layout was sketched out.

\begin{figure}[h]
\centering
\includegraphics[width=0.5\textwidth]{figures/RevisedControl} 
\caption{The new controller based on the paper play test}\label{fig:PPtest4}
\end{figure}

Now the controller would initially display only the controls used to plan the turn, such as move and interact, as can be seen in Figure \ref{fig:PPtest4}. However, an area on the smartphone would indicate how many map pieces were collected. This area would also have a button that could be pressed to overlay the map on top of the controls, so the player could inspect what their map pieces showed. The same button could then be pressed to close the map overlay and, again, show the controls. Additionally, when a player had their map open and it became their turn, the map could automatically close as a signal that it was their turn.

The map overlay was not implemented due to time constraints. However the counting of map pieces was implemented.