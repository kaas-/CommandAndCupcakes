\pdfbookmark[0]{English title page}{label:titlepage_en}
\aautitlepage{%
  \englishprojectinfo{
    Treasure Hunt %title
  }{%
    Audio-Visual Experiments %theme
  }{%
    Fall Semester 2016 %project period
  }{%
    MTA16537 % project group
  }{%
    %list of group members
	Emilie Lind Damkjær\\
	Liv Arleth\\
	Markus Kristian Ørbæk Bertelsen\\
	Margarita Kaljuvee\\   
    Peter Kejser Jensen\\ 
    Rasmus Kaasgaard Christiansen
  }{%
    %list of supervisors
    Martin Kibsgaard
  }{%
    6 % number of printed copies
  }{%
    \today % date of completion
  }%
}{%department and address
  \textbf{Medialogy}\\
  Aalborg University\\
  \href{http://www.aau.dk}{http://www.aau.dk}
}{% the abstract
New concepts can take advantage of using what people already have. The gaming platform AirConsole did, by using smartphones as controller and a computer, which allows for easy local multiplayer. In some games, this means that players have to switch their attention between the screens. This requires game design that directs attention. This raises the question "How can the player’s attention be directed between a shared screen and a personal hand-held screen when there is a need to change their visual focus between the two in a local multiplayer game?" To answer this question, Treasure Hunt was developed. This AirConsole game consists of four versions, each using a form of attention directing stimulus. 24 people played the game, and via questions and face recognising software gave results regarding each attention directing methods’ effectiveness. Visual stimulus on the computer screen worked most effectively, with tactile stimulus in the smartphone coming a close second. Visual stimulus on the smartphone did not perform well. These results can help designers to make systems that need to direct attention between two screens. Combining the tested methods could have increased effectiveness, but require further research.
}