\chapter{Introduction}\label{ch:intro}
In this modern day it is a common sight to see almost everyone with a smartphone either in use or at least near their person at almost all times. This can be used advantageously to design around when coming up with a new concept. The gaming platform AirConsole did exactly this, creating essentially a videogame console for local multiplayer using what people already had in their pockets as the controller and a regular computer as the console itself. Creatively, this allows the controls to work and look in many different ways, given the controller consists entirely of programmable touchscreen.

However, this setup also raises some interesting questions. The players has to divide their attention between two screens: their own smartphone, and the computer monitor shared between all players in the room. Is there any way to make use of the way people react to stimuli? What stimuli are more effective to direct attention between the two screens?

These questions are what this project seeks to answer. It does so by first investigating theory on what attention actually is, and how to manipulate and measure it. From this theory, an experiment is established in an attempt to find methods that successfully direct attention between the two screens. The experiment consists of a small game with multiple versions where different attention directing methods are included in each one.