\chapter{Treasure Hunt: a dual-screen testing tool}\label{ch:game}
As the focus of this project lies on directing attention between two screens, a game has been developed for the purpose of testing the problem formulation. The game is an AirConsole game, which means that players will need to direct their attention at the main screen in certain situations, and at their phone screen in others. Therefore, the game serves as a tool to test the best way to direct a player's attention towards a certain screen.

The game is a turn-based treasure hunting game for four players. Each player controls a pirate character which they use to walk around on a treasure island to look for map pieces and copy them from the other players. When a player has found all nine map pieces, they win the game.

The game is designed in a way so that players will have too look at the main screen in certain situations, and on their phone screen in others. This decision was made so that the game works as a testing tool to investigate how to direct players' attention between two screens. Furthermore, the game is turn-based, rather than real-time, as the AirConsole format has an inherent latency issue. Making the game turn-based reduces the consequences of a delay between player input and game reaction time.

\section{Gameplay}
Each player has a turn in a random order---once all players have had their turn, this order is scrambled. On a player's turn, they must plan two actions within a limited time frame, which will be executed in the player specified order. If the time runs out, the actions they have planned so far will be executed, so if only one action has been planned only that one will be executed. There are two possible actions: moving in one of four directions and interacting with an object. The actions are planned by tapping on the corresponding button in the order the player would like them to execute in. The \textit{interact} action makes the player's pirate searches the object at their current position. This is one way to find map pieces. 

If a player ends their turn on a tile which another player already occupies then combat will be initiated between the two. A minigame will start on the two players' smartphones. The player who wins the fight is shown a "YOU WON" message on their smartphone, while the player who loses is shown a "YOU LOST" message. Whoever wins the fight will then copy a map piece from the loser if the loser is in possession of a map piece that the winner does not already have. If the loser is not in possession of any map pieces that the winner is currently missing, then no new map piece is gained. This was implemented to avoid players from copying the same map pieces over and over in order to win, and also requiring all map pieces to be found from the objects by someone.

The random turn-order was decided upon based on the problem statement regarding directing attention. This would be a way to test different ways to direct the players attention without them growing accustomed to the turn-order. The time limit is to increase the pace of the game. However, if a player is last in a turn order, then they cannot become the first player. This was included so that a player could not effectively get two turns in a row.

The reason the combat minigame was implemented is to force the players to redirect attention when it is not necessarily their turn.

\section{Interfaces}
The game has two different types of screens: a shared main screen and a personal smartphone screen for each player. This is a feature of the AirConsole platform.

On the main screen is the game arena which is a tropical island. This island has a visible grid pattern on it to indicate to the players where they can walk and where the boundaries of the arena are. Interactable objects such as palm trees, ferns and skulls are placed inside of the corners of these grids and they can be interacted with if a player character is on the same tile. The island is square and it is viewed from an isometric perspective. 

The grid based arena was chosen to make it easier for the players to navigate the map. The objects are also placed where the players find it most intuitive to be able to interact with them when on the same tile. Therefore, skulls and ferns populate the middle and lower border of the island, as these do not cover much area of the screen. The palm trees cover a lot more of the screen, and are placed around the upper border, so they do not cover any of the other tiles. The game has a cartoonish style with high colour saturation in order to draw the players' attention. This was done so that players' will be easily find their own character and plan their  turn.

\begin{figure}
	\centering
	\includegraphics[width = 0.5\textwidth]{figures/MainScreen}
	\caption{The main screen}\label{fig:main_screen}
\end{figure}

The smartphone screen displays the colour of the player, the amount of map pieces the player has, the slots for the planned actions, five action buttons the player can tap during their turn, and the button to execute their turn. This is shown in Figure \ref{fig:phone_screen}.

\begin{figure}
	\centering
	\includegraphics[width =0.5\textwidth]{figures/PhoneScreen}
	\caption{The phone screen of the yellow player during their turn}\label{fig:phone_screen}
\end{figure}

The movement buttons are arranged in a two-by-two pattern, and the interact button is placed beside them to the right. Two slots are displayed at the top of the screen. This is where any planned actions will appear. The button with a check mark, seen on the right side of the screen, can be tapped to execute the planned actions. On the top-left of the smartphone screen is an icon with a little grid-map on it and a number underneath it. The leftmost number is the number of map pieces the player is in possession of, and the rightmost number is the number of pieces needed. Lastly in the top-right corner, a timer is placed which displays the remaining time of that player's turn. The interface is designed in such a way that the phone screen does not become cluttered with features and all of the elements are simple in design and easily recognisable.

\section{Attention-directing stimuli}
Four versions were made of this game. Three of them utilise different methods to direct attention, while the fourth one has no attention-directing stimuli. The stimuli are given to individual players when their turn starts and when combat in which they participate begins. The implemented versions are:

\begin{enumerate}[label=\Alph*)]
\item No stimuli
\item Tactile stimulus through vibrations in the smartphone
\item Visual stimulus through splash screens on the main screen
\item Visual stimulus through splash screens on the smartphones
\end{enumerate}

\subsection{No stimuli}\label{sub:no_feedback}
Labelled \textit{A} in the experiment, it is used as a control version to analyse how people will distribute their attention when the system does not give any cues on where players are supposed to look. With this version, it is possible to investigate if certain attention cues work at all, by comparing the results of the version with said attention cues to the data from this version.

\subsection{Tactile stimulus}\label{sub:tactile_feedback}
Labelled \textit{B} in the experiment, this version of the game uses vibrations in the smartphones. This vibration is one continuous pulse lasting 200 milliseconds.

\subsection{Visual stimulus on the main screen}\label{sub:visual_main}
Labelled \textit{C} in the experiment, this version of the game displays splash screens on the main screen as an attention cue. A splash screen is shown every time a player begins their turn, which corresponds to the colour of that player's controller and pirate. An example of this can be seen in Figure \ref{fig:redturn}.

A splash screen is also shown on the main screen every time combat begins between any two players. This splash screen can be seen in Figure \ref{fig:fight_splash}.

\begin{figure}[h!]
	\centering
	\includegraphics[width=0.5\textwidth]{figures/redturn.png}
	\caption{Splash screen for red player's turn}\label{fig:redturn}
\end{figure}

\begin{figure}[h!]
	\centering
	\includegraphics[width=0.5\textwidth]{figures/getready.png}
	\caption{Splash screen for combat}\label{fig:fight_splash}
\end{figure}

\subsection{Visual stimulus on the smartphones}\label{sub:visual_smartphone}
Labelled \textit{D} in the experiment, this version of the game uses the same splash screens as in version \textit{C}. However, the splash screens are used only for switching a player's turn. These splash screens are shown on the smartphones, and a splash screen is only shown to the player that is currently starting their turn. This means that in case it is the red player's turn, only said player will have a splash screen displayed on the smartphone.

In order to test how well these methods work, the following chapter will describe how testing was designed and performed.