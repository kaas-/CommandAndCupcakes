\chapter{The Experiment}\label{ch:experiment}
\section{The Game Description}
The game is a treasure hunting game for multiple players. Each player controls a pirate character which they then use to walk around on the treasure island, to look for map pieces or to steal them from the other players. When a player has found enough map pieces to be able to deduce where the treasure is hidden they can then walk over to that spot and dig the treasure up. They have to be smart about it since the other players can attack them while they are digging. The first player to reach the treasure wins. 


\subsection{Gameplay}
Each player have a turn in a randomised order, where they each can plan two actions for that turn. They have a [few] seconds to do this. The actions are the following: Move in a direction, interact and dig. The actions are planned by clicking on the corresponding button in the order you would like them to execute in. There is a movement option for each of the four directions. The interact action makes the players pirate interact with an object on the tile it is on. This is one way to find map pieces. If a player actively moves onto a tile another player already occupies then an attack will be initiated. A minigame will start on the two players smartphone. The winner stays on the tile and the loser gets pushed back two tiles in a valid direction going clockwise. The loser also loses their first action in their next turn on account of them being “stunned”. If another player occupies this tile no attack will be initiated since the move was not active. The winner also gets one of the loser’s map pieces. The dig action is a bit different since this takes two actions and until it is the player’s turn again to complete. This gives the other players an opportunity to disrupt the dig by attacking the player successfully. When the dig is finished, and if it was in the correct space the digging player wins the game. 


\subsection{Interfaces}
The game has two different type of screens. A shared main screen and a personal smartphone screen for each player. 
On the main screen is the game arena which is in the shape of a tropical island. This island has a visible grid pattern on it to indicate to the players where they can walk and how far they can walk. Interactable objects such as palm trees, ferns and skulls are placed inside of the corners of these grids and they can be interacted with if a player character is on the same grid space. The island is square and it is viewed at an isometric angle. Additionally there is also a scoreboard on the main screen displaying the number of map pieces each player has.
\todo{add picture of the main screen interface}

The smartphone screens displays the colour of the player, six buttons you can tab during your turn to plan the turn. 

\todo{add picture of smartphone interface}

As can be seen on figure XX the buttons are arranged in a two-by-three pattern with the movement buttons slightly closer than the other two buttons. Two empty spaces at the top is where the planned actions are shown. The button with a checkmark is at the upper right corner and is the button that needs to be tapped to implement the actions the player planned. The last thing on the smartphone screen is a button with a little grid-map on it and a number underneath it. The left number is the number of map pieces the player is in possession of and the right number is the number of pieces needed. If the button with the grid-map is tapped the treasure map will appear on the smartphone screen. Another tap and it is gone. 

