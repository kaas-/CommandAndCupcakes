\chapter{The experiment}\label{ch:experiment}
The game described in Chapter \ref{ch:game} serves as a tool to test the problem formulation:

\begin{changemargin}{1cm}{1cm}
\textbf{How can the player’s attention be directed between a shared screen and a personal hand-held screen when there is a need to change their visual focus between the two in a local multiplayer game?}
\end{changemargin}

Since the focus of this problem formulation is on directing attention between two screen in a setup such as that of the AirConsole, the experiment described in this chapter sought to discover which method is best fit for directing the players' attention. To do this, test participants were subjected to the four different iterations of the game, which, as mentioned in Chapter \ref{ch:game}, utilise four different methods of directing attention. The effect on the player's attention were measured through a combination of self-report questions and gaze tracking. This chapter describes the experiment in further detail.

\section{Metrics}
In order to test the problem formulation, a combination of several metrics were measured. These metrics may be split into two categories: performance and user experience metrics.

\subsection{User experience metrics}
After playing an iteration of the game, test participants were asked to fill out a questionnaire. The answers from this questionnaire serve as user experience metrics. As this is self-reported data from the participants, it is fit for measuring the users' own perception, but less so for measuring their tangible performance. However, as the users' own experience is still relevant to the problem formulation, this experiment made use of self-report questionnaires. The questions in the questionnaire were as follows:

\begin{enumerate}
	\item To which degree do you agree with the following statements?
	\begin{enumerate}
		\item It was clear when it was my turn to act
		\item It was clear when I was in combat
		\item It was clear when I was supposed to look at the phone screen
		\item It was clear when I should look at the main screen
		\item It was easy for me to switch my attention between the two screens
		\item It was easy to find what I was looking for when switching my gaze between the screens
		\item I was distracted by the other players' phones
	\end{enumerate}
	\item What did you think of the dual-screen setup? What did you enjoy about it? What did you find confusing?
	\item Do you have any additional notes?
\end{enumerate}

The rating scale statements (that is, the statements presented under question 1) provide seven possible answers: Strongly disagree, disagree, somewhat disagree, neutral, somewhat agree, agree, and strongly agree. The inclusion of the \textit{neutral} option allows participants who neither agree nor disagree with a statement to pick a fitting option, rather than being forced to decide whether they agree or disagree, which may damage the validity of the test. The inclusion of the \textit{slightly agree} and \textit{slightly disagree} options allow participants who feel that they only partially agree with a statement to pick a side without sounding too opinionated, which may dissuade some from picking the completely neutral reason even though they agree more than they disagree.

\subsection{Performance metrics}
Performance metrics measure tangible results observed throughout the experiment. While playing the game, players were recorded both with a video camera and a Kinect 2, which logged when the players looked at the main screen, and when they looked away. The game logged when it was a new player's turn. By comparing these time stamps, it was possible to measure how much time passed from the moment it became a player's turn, to the moment they looked at their smart phone. The time stamps logged by the game also allowed for the measurement of performance time, namely the time it takes for a participant to complete their turn. In addition to this, error rates were also measured; for each turn, it was noted whether a participant looked at their screen at all, and whether they completed their turn before the given time ran out. 

\section{Setup}
Each test included four participants. Apart from these, three people were present to facilitate the test:

\begin{itemize}
	\item A \textbf{test conductor}, who instructed the participants in the terms of the test
	\item A \textbf{note taker}, who wrote down anything of note
	\item A \textbf{technician}, who made sure that both the Kinect 2 and the game were running smoothly, and that the camera was on.
\end{itemize}

Apart from the test conductor, the people facilitating the test remained as silent as possible throughout the test in order to not interfere with its results. The following materials are used for the purpose of this test:

\begin{itemize}
	\item A television screen
	\item A computer that runs the game, connected to the TV screen through an HDMI cable
	\item A Kinect 2
	\item A computer that runs the Kinect, connected to it through a USB 3.0 cable
	\item 4 smartphones
	\item A video camera
	\item One consent form for each participant
	\item Four questionnaires for each participant
	\item Four different builds of the game
\end{itemize}

These materials were set up as shown in Figure \ref{fig:test_setup}. The four participants were seated (not necessarily on a sofa) in front of a large television screen, to which a computer running the game was connected through an HDMI cable. This way, the main screen for all four participants was the television screen. Underneath this screen was a Kinect 2, which was aimed at the participants while connected to another computer through a USB 3.0 cable. In front of each participant was a smartphone, which was connected to the game.

\begin{figure}[h!]
	\centering
	\includegraphics[width=\textwidth]{figures/test_setup.png}
	\caption{Test setup, showing the television screen (A), the Kinect 2 (B), a computer running the game (C), a computer running the Kinect 2 (D), seating for participants (E), and 4 smartphones (F)}\label{fig:test_setup}
\end{figure}

\section{Method}
Once participants were seated, the test conductor explained the test procedure, as well as the basics of the game. Once every participant was ready, the game began, and participants played freely until a the game ended or 10 minutes had passed, whichever came first. While the game was ongoing, the participants were filmed with the video camera, and data about their gaze was recorded with the Kinect 2. Each group of participants played through four iterations of the game, in a randomised order. The four iterations are, as mentioned in Chapter \ref{ch:game}, fo

\begin{enumerate}[label=\Alph*)]
	\item 
	\item u hottie
\end{enumerate}

After the game has ended or the given time is up, participants are given a questionaire, with a mixture of qualitative questions and Likert scale questions. These questions are as follows:
\section{Results}

\section{Analysis and Conclusion}