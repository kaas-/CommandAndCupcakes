\chapter{The experiment}\label{ch:experiment}
The game described in Chapter \ref{ch:game} serves as a tool to test the problem formulation:

\begin{changemargin}{1cm}{1cm}
\textbf{How can the player’s attention be directed between a shared screen and a personal hand-held screen when there is a need to change their visual focus between the two in a local multiplayer game?}
\end{changemargin}

Since the focus of this problem formulation is on directing attention between two screens in a setup such as that of the AirConsole, the experiment described in this chapter sought to discover which method is best fit for directing the players' attention. To do this, test participants were subjected to the four different iterations of the game, which, as mentioned in Chapter \ref{ch:game}, utilise four different methods of directing attention. The effect on the player's attention were measured through a combination of self-report questions and gaze tracking. This chapter describes the experiment in further detail.

\section{Metrics}
In order to test the problem formulation, a combination of several metrics were measured. These metrics may be split into two categories: performance and user experience metrics.

\subsection{User experience metrics}\label{subsec:user_experience_metrics}
After playing an iteration of the game, test participants were asked to fill out a questionnaire. The answers from this questionnaire serve as user experience metrics. As this is self-reported data from the participants, it is fit for measuring the users' own perception, but less so for measuring their concrete performance. However, as the users' own experience is still relevant to the problem formulation, this experiment made use of self-report questionnaires. The questions in the questionnaire were as follows:

\begin{enumerate}
	\item To which degree do you agree with the following statements?
	\begin{enumerate}
		\item It was clear when it was my turn to act
		\item It was clear when I was in combat
		\item It was clear when I was supposed to look at the phone screen
		\item It was clear when I should look at the main screen
		\item It was easy for me to switch my attention between the two screens
		\item It was easy to find what I was looking for when switching my gaze between the screens
		\item I was distracted by the other players' phones
	\end{enumerate}
	\item What did you think of the dual-screen setup? What did you enjoy about it? What did you find confusing?
	\item Do you have any additional notes?
\end{enumerate}

The rating scale statements (that is, the statements presented under question 1) provide seven possible answers: Strongly disagree, disagree, somewhat disagree, neutral, somewhat agree, agree, and strongly agree. The inclusion of the \textit{neutral} option allows participants who neither agree nor disagree with a statement to pick a fitting option, rather than being forced to decide whether they agree or disagree, which may damage the validity of the test. The inclusion of the \textit{slightly agree} and \textit{slightly disagree} options allow participants who feel that they only partially agree with a statement to pick a side without sounding too opinionated, which may dissuade some from picking the completely neutral option in these cases.

\subsection{Performance metrics}\label{subsec:performance_metrics}
Performance metrics measure concrete results observed throughout the experiment. While playing the game, players were recorded both with a video camera and a Kinect 2 which logged when the players looked at the main screen, and when they looked away. The game logged when it was a new player's turn. By comparing these time stamps, it was possible to measure how much time passed from the moment it became a player's turn, to the moment they looked at their smart phone. The time stamps logged by the game also allowed for the measurement of performance time, namely the time it takes for a participant to complete their turn. In addition to this, error rates were also measured; for each turn, it was noted whether a participant looked at their screen at all, and whether they completed their turn before the given time ran out. 

\section{Setup}
Each test included four participants. Apart from these, three people were present to facilitate the test:

\begin{itemize}
	\item A \textbf{test conductor}, who instructed the participants in the terms of the test
	\item A \textbf{note taker}, who wrote down anything of note
	\item A \textbf{technician}, who made sure that both the Kinect 2 and the game were running smoothly, and that the camera was on.
\end{itemize}

Apart from the test conductor, the people facilitating the test remained as silent as possible throughout the test in order to not interfere with its results. The following materials were used for the purpose of this test:

\begin{itemize}
	\item A television screen
	\item A computer that runs the game, connected to the TV screen through an HDMI cable
	\item A Kinect 2
	\item A computer that runs the Kinect, connected to it through a USB 3.0 cable
	\item 4 smartphones
	\item A video camera
	\item One consent form for each participant
	\item Four questionnaires for each participant
	\item Four different builds of the game
\end{itemize}

These materials were set up as shown in Figure \ref{fig:test_setup}. The four participants were seated in front of a large television screen, to which a computer running the game was connected through an HDMI cable. This way, the main screen for all four participants was the television screen. Underneath this screen was a Kinect 2 which was aimed at the participants while connected to another computer through a USB 3.0 cable. In front of each participant was a smartphone, which was connected to the game.

\begin{figure}[h!]
	\centering
	\includegraphics[width=\textwidth]{figures/test_setup.png}
	\caption{Test setup, showing the television screen (A), the Kinect 2 (B), a computer running the game (C), a computer running the Kinect 2 (D), seating for participants (E), and 4 smartphones (F)}\label{fig:test_setup}
\end{figure}

\section{Method}
Once participants were seated, the test conductor explained the test procedure as well as the basics of the game. Once every participant was ready, the game began, and participants played freely until a the game ended or 10 minutes had passed, whichever came first. While the game was ongoing, the participants were filmed with the video camera, and data about their gaze was recorded with the Kinect 2. Each group of participants played through four iterations of the game, in a randomised order. The four iterations, as mentioned in Chapter \ref{ch:game}, utilise four different methods for directing the player's attention to their phone screen when it becomes their turn, as well as when they are engaged in combat. These four methods are:

\begin{enumerate}[label=\Alph*)]
	\item No feedback
	\item Tactile feedback (phone vibration
	\item Visual feedback (warning image on main screen)
	\item Visual feedback (warning image on phone screen)
\end{enumerate}

After a group of four participants had tried an iteration of the game, every participant was asked to fill out the questionnaire described in Section \ref{subsec:user_experience_metrics}. While the game was ongoing, the Kinect logged whether players looked at or away from the main screen, and the game logged whose turn it was at any given time. The note taker noted down each player's error rates as described in Section \ref{subsec:performance_metrics}.

\section{Results}
\subsection{User experience results}
After having played a version of the game, each participants as asked to fill out the questionnaire described in section \ref{subsec:user_experience_metrics}. The rate at which participants agreed with the statements given in question 1 will be described in this section. The answers given in statement 1a are illustrated in Figure \ref{fig:questionnaire_a}. From this figure, it is clear that participants most strongly agreed with the notion that it was clear when it was their turn to act when given tactile stimulus, or visual stimulus on the main screen. They found it most unclear when given nu stimulus at all.

\begin{figure}[h!]
	\centering
	\includegraphics[scale=1]{figures/questionnaire_a.png}
	\caption{Answers for the question "it was clear when it was my turn to act" for each of the four versions}\label{fig:questionnaire_a}
\end{figure}

Figure \ref{fig:questionnaire_b} shows that it was somewhat clearer to participants when they were in combat when given no stimulus, but similarly to the previous question, participants mostly found it clear when given tactile stimulus or visual stimulus on the main screen.

\begin{figure}[h!]
	\centering
	\includegraphics[scale=1]{figures/questionnaire_b.png}
	\caption{Answers for the question "it was clear when I was in combat" for each of the four versions}\label{fig:questionnaire_b}
\end{figure}

Similar results are seen in Figure \ref{fig:questionnaire_c}, where players were least in doubt regarding when they were supposed to look at the phone screen when given main screen or tactile stimuli. However, the answers seen in Figure \ref{fig:questionnaire_d}, players were generally neutral regarding whether it was clear when they should be looking at the main screen. The version with tactile feedback stands out in this case as the version where most people found it clear when they should look at the main screen.

\begin{figure}[h!]
	\centering
	\includegraphics[scale=1]{figures/questionnaire_c.png}
	\caption{Answers for the question "it was clear when I was supposed to look at the phone screen" for each of the four versions}\label{fig:questionnaire_c}
\end{figure}

\begin{figure}[h!]
	\centering
	\includegraphics[scale=1]{figures/questionnaire_d.png}
	\caption{}\label{fig:questionnaire_d}
\end{figure}

Figure \ref{fig:questionnaire_e} illustrates to which degree participants found it easy to switch their attention between the two screens in the four different versions of the game. The results here show that people especially found it easy in the versions with tactile and main screen stimuli.

\begin{figure}[h!]
	\centering
	\includegraphics[scale=1]{figures/questionnaire_e.png}
	\caption{}\label{fig:questionnaire_e}
\end{figure}

Figure \ref{fig:questionnaire_f} shows that players generally found it easy to find what they were looking for when switching their gaze between the two screens. These results stand out compared to those of the previous questions, as they do not seem to be affected as much by the stimulus given to the participants when it was their turn, and when they are engaged in combat; even the versions with no stimulus, or visual stimulus on the phone screen, got more positive results on this question.

\begin{figure}[h!]
	\centering
	\includegraphics[scale=1]{figures/questionnaire_f.png}
	\caption{}\label{fig:questionnaire_f}
\end{figure}

\subsection{Performance results}
As mentioned in section \ref{subsec:performance_metrics}, two logs were generated throughout the test: one noting time stamps for different events on the game, and one noting data from the kinect regarding whether players looked at or away from the main screen. By comparing the two, it was possible to measure the amount of time it took for each player to look at their phone after it became their turn, as well as after they were engaged in combat. This reaction time was noted for each of these events, and an average reaction time was calculated for each of the four versions of the game. As can be seen in Figure \ref{fig:reaction_time}, participants switched their visual attention fastest when given tactile stimulus, at an average of 0,78 seconds. Players switched their visual attention slowest when given visual stimulus on the main screen.

\begin{figure}[h!]
	\centering
	\includegraphics[scale=1]{figures/reaction_time.png}
	\caption{Average reaction time for the four versions of the game}\label{fig:reaction_time}
\end{figure}

It is, however, significant to note that in some cases, test participants failed to look at the phone screen at all during their turn or a combat they were a part of, and in other cases, participants were already looking at their phone by the time it became their turn or they engaged in combat. In these cases, it was impossible to note a reaction time. Instead, Figure \ref{fig:looked} illustrates the rate at which these two circumstances occur, for each of the four versions. As the figure shows, these cases occurred most frequently when players were given no indication of when it was their turn, and least so when they were given visual stimulus on the main screen. 

\begin{figure}[h!]
	\centering
	\includegraphics[scale=1]{figures/graph_looked.png}
	\caption{The rate at which participants failed to look at their phone screen, or were already looking at it, for each of the four versions}\label{fig:looked}
\end{figure}

\section{Analysis and Conclusion}
When looking at the performance metric results, it is important to note that even though the game version with visual stimulus caused test participants to perform the worst with regards to reaction time, it is the version which caused fewest cases where participants were already looking at the screen to wait for their turn, whereas the version with no indication of whose turn it is caused faster reactions from participants, but many would constantly look at their phone screen, even if it was not their turn. This is possibly because they were anticipating their turn, but since they would be given no indication of when their turn would start, they had to check in with their phone screen more often. \todo{move this somewhere reasonable}