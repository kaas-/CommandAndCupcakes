\chapter{Conclusion}\label{ch:conclusion}
This project aimed to investigate the following problem statement:

\begin{changemargin}{1cm}{1cm}
\textbf{How can the player’s attention be directed between a shared screen and a personal hand-held screen when there is a need to change their visual focus between the two in a local multiplayer game?}
\end{changemargin}

In order to answer this problem statement, three research questions were posed. These questions were all areas which we sought to explore in the final experiment:

\begin{itemize}
\item What effect does different types of sensory feedback have on the players' attention on a two-screen platform such as the AirConsole?
\item If given no indication of where to look, are players more likely to look at the phone screen while it is irrelevant to them?
\item If given no indication of where to look, are players more likely to fail to look at their phone screens while it is relevant to them?
\end{itemize}

In order to answer these questions, a game was made for the AirConsole platform. This game served as a testing tool to investigate the effects of different types of stimuli on the players’ attention in a dual-screen setup, such as that of the AirConsole. Four versions of the game were made: a control version, a version with tactile stimulus, a version with visual stimulus on the main screen, and a version with visual stimulus on the phone screens. These four versions were compared in an experience which combined various metrics to answer the posed questions.

Based on the results gathered from the experiment, the answer to the first question is as follows: tactile stimulus generally causes participants to have a faster reaction time when it comes to switching their attention from the main screen to the visual screen. It does, however, also yield a relatively higher rate of cases where participants look at their phone screen at times when it is irrelevant to them, as well as cases where participants failed to look at their phone screens at all. User experience tests show that tactile feedback performs relatively well when it comes to communicating to the players where they should aim their attention, although not as well as visual feedback on the main screen. This version, while yielding slower reactions from players, performs well when it comes to leading their gaze to the phone screen in the right situations, as well as in every aspect of the user experience tests. The version with visual stimulus on the phone screen does not have a noticeable effect on players’ reaction time, and fails to make it clear to the users where they should aim their attention.

As for the second question, results from the experiment showed that the version where players are given no indication of where to look is also the version which caused the most cases where people were already looking at their phones by the time it became their turn. Since the phone screen is irrelevant to the players while it is not their turn, it can be concluded that players are more likely to look at their phone screen in situations where it is irrelevant to them when given no indication of where to look.

The aforementioned version of the game was also the version which caused more cases where players would fail to look at their phone at all during their turn, thus missing their opportunity to act. Thus, the answer to the third question is that players are more likely to fail to look at their phone screens while it is relevant to them if given no indication of where to look.

\textbf{In conclusion}, results from the experiment indicated that in order to direct players’ attention between a shared screen and a personal hand-held screen, some stimulus needs to be given to the players, as a game with no such stimulus failed to properly direct the test participants’ attention. Having obtained the results from the evaluation, it can be stated that the visual stimulus on the main screen worked best for the aforementioned purpose. It could, however, be combined with other types of stimulus, such as tactile feedback, as it proved to guide the players’ attention faster in certain circumstances. This knowledge, combined with other ways to direct attention stated in the background research, could possibly assist in guiding the players attention in such a way that is desired by the game developer.