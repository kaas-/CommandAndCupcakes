\chapter{Introduction}\label{ch:introduction}
While designing any piece of technology that will be utilised by an individual, it is important to take into consideration the user requirements. This is especially relevant during game development, since the quality of player-game interaction is what will keep the game relevant to the user. One of the most important aspects of this kind of interaction is engagement, which Charlton and Danford (2010) \cite{Charlton2010} define as ‘‘a high degree of positive involvement in computer usage’’. 

To measure engagement and define certain factors that influence it, one can utilise flow theory. As stated by Hamari and Koivisto, "The flow state has been widely used to describe an optimal experience
characterized as a state of being fully focused and engaged in
an activity" \cite{Hamari2014}. In order to identify the state of flow more clearly, nine dimensions of flow can be used. The fifth dimension, which is defined as “focusing on the task at hand”, might be especially interesting when developing a game on a platform that utilises two screens at once, such as \textit{AirConsole}.

AirConsole is a browser-based local multiplayer game platform --- “local” in this case meaning that all players are in the same room. A computer serves as the main screen which is accessible to all players. In addition, each player uses their personal smartphone as the means to interact with the game, and as an individual interface. This presents some unique options for gamemakers, since this platform provides two screens for each player to work with instead of the one screen most platforms provide. However, it can also pose specific challenges, since the user is required to pay attention not only to the shared computer screen, but also to their personal smartphone screen. Usually when one shifts their attention between two things, the eyes move accordingly in order to have the object of interest portrayed on the forea. Constantly moving the eyes from one position to another might make a player miss some important information on one of the screens, which may lead to confusion during gameplay. Furthermore, the attention of an individual may be broken by exogenous events which are not related to the player-game interaction. Since AirConsole is a local multiplayer platform, the attention span of a player may be even shorter due to the distractions from other players. Therefore, if the game developer fails to consider the imposed challenges, the overall impression from the gameplay might be ruined. This leads to a problem formulation:

\begin{changemargin}{1cm}{1cm}
\textbf{How can the player’s attention be maintained when there is a need to change the point of view between two screens in a local multiplayer game?}
\end{changemargin}

This problem formulation caused some additional questions to arise. These questions will be addressed throughout the project, and seek to ultimately answer the problem formulation:
\begin{itemize}
\item What effect does different types of sensory feedback have on the players' attention on a two-screen platform such as the AirConsole?
\item If given no indication of where to look, which of the two screens will the players most likely look at?
\end{itemize}

Before an experiment addressing the problem formulation can be conducted, it is essential to have some knowledge about attention. Therefore, the next chapter will examine attention, the terms used to describe it, and methods which can be used to measure it.