\chapter{Further research}\label{sec:future_research}
This chapter describes some of the possible directions in which this research can be taken to investigate the subject further. 

The methods used in this project to direct attention are based on the research that is described in Chapter \ref{ch:attention}. These methods are used to direct attention in a situation where the users need to alternate their attention between two screens and are given stimuli to indicate which screen should receive full attention at a given time. These methods are based on the theory of bottom-up attention, as the desired outcome of a stimulus is that the user reacts by switching their attention from one screen to the other.

In this project, the tactile stimulus is a consistent continuous vibration. To further investigate how vibration should be used as a good cue for directing attention, the pattern of vibration could be changed. For example, vibrations could vary in intensity depending on what is happening. Alternatively, a tactile cue could consist of a vibration which changes in intensity over time. Further research could look into how different lengths and intensities of vibrations affects users' attention.

The visual stimuli on the main screen and on the phone covered the whole screen in this experiment. This meant that when a player looked at the screen where a cue was displayed, the player would see the cue. However, it is unknown if covering the whole screen is the most effective way of using this type of visual stimulus. Variations in position, size, and duration could be changed to investigate how visual stimuli could be used more effectively to get the players to look at the other screen when needed.

The stimuli used in this experiment do not cover all possible stimuli that could be used to direct attention. Audio stimuli from either speakers near the main screen or from the speaker in the smartphone could be investigated to see how effective this method would be. In a setting such as the one in this project, audio could potentially distract other players, especially if the audio stimuli comes from the smartphones. It was considered to use this as one of the tested methods, but because of the aforementioned possibility of distraction and time constraints, this was not implemented and tested. Therefore, further investigation is required to conclude if audio could potentially be useful as an attention-directing stimulus in local multiplayer scenarios.

Additionally, this project tested the methods of stimuli in isolation from each other, to conclude which methods are effective in and of themselves. With effective methods established, further research could be done to see how their effectiveness is affected by combining different methods with each other. One way could be with a hypothesis based on the results from the test of this project, such as: The combination of main screen and tactile stimuli decreases the reaction time of players compared to players who are only given one type of stimulus.