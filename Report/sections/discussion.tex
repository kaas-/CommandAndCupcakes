\chapter{Future Research}\label{sec:future_research}
This chapter describes some of the possible directions in which this research can be taken, to investigate the subject further. 

The methods used in this project to direct attention is based on the research that is described in Chapter\ref{ch:attention}. These methods tried directing attention in a situation, where the user have alternating attention between the two screens and are given feedback to which screen should receive full attention at a given time. These methods are based on the theory of bottom-up attention, as the desired outcome of a cue is the player reacts by switches from one screen to the another.

The visual and tactile feedback used in the experiment are one variation out of many. In this project, the feedback are flat pulses in effect for a consistent amount of time. Specifically with the tactile feedback, this results in evenly distributed amounts of vibration during the cue. To further investigate how vibration should be used as a good cue for directing attention, the vibration could be changed. This means that the flat pulse used in the experiment would be used as a baseline. Compared to this baseline could be variations in strength, length and pattern of the vibration. For example could the intensity of the vibration follow a sinus curve instead of the flat line implemented in this project. This would clarify how vibration can be used effectively as a attention-directing cue.

The visual feedback on the main screen and on the phone covered the whole screen in this experiment. This meant that when a player looks at the screen where a cue currently is displayed, the player will see the cue. However it is unknown if covering the whole screen is the most effective way of using this type of visual feedback. Variations in position, size, and time could be changed, to investigate how visual feedback could be used more effectively to get the players to look at the other screen when needed.

However, this project only used visual and tactile feedback to direct attention between the two screens. Other methods could be used since AirConsole is using smartphones as controllers. Audio feedback from either speakers near the main screen or from the speaker in the smartphone could be investigated to see how effective this method would be. In a setting such as the one in this project, audio could potentially distract other players, especially if the audio feedback comes from the smartphones. This project considered using this as one of the tested methods, but because of the before mentioned possibility of distraction and time constraints, this was not implemented and testing. Therefore, further investigation is required to conclude if audio could potentially be useful as attention-directing feedback in local multiplayer scenarios.

Additionally, this project tested the methods of feedback in isolation from each other, to conclude which methods are effective in and of themselves. With effective methods established, further research could be made to see how their effectiveness is affected by combining different methods with each other.