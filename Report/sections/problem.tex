\chapter{Problem statement}\label{ch:problem}
While designing any technological artefact that will be utilised by an individual, it is important to take the user's needs into consideration. This is no less significant if the artefact in question is a game --- in this case, the user needs in question may be related to how enjoyable the game is for the user to play, and how engaging it is. Engagement is described by Charlton and Danford (2010) \cite{Charlton2010} as ‘‘a high degree of positive involvement in computer usage’’. 

To measure engagement and define certain factors that influence it, flow theory can be utilised. As stated by Hamari and Koivisto, ‘‘The flow state has been widely used to describe an optimal experience characterized as a state of being fully focused and engaged in an activity’’ \cite{Hamari2014}. There are several ways to describe the state of flow. In the nine dimensions of flow theory described by Hamari and Koivisto, the fifth dimension, which is ‘‘focusing on the task at hand’’, is especially interesting when developing a game on a platform that utilises two screens at once, such as \textit{AirConsole}.

AirConsole is a browser-based local multiplayer game platform --- “local” in this case meaning that all players are in the same room. A computer serves as the main screen which is accessible to all players. In addition, each player uses their personal smartphone as the means to interact with the game as an individual interface. This presents some unique options for game developers, since this platform provides two screens for each player to work with instead of the one screen most platforms provide. However, it can also pose specific challenges, since the user is required to pay attention not only to the shared computer screen, but also to their personal smartphone screen. Usually, when a person shifts their attention between two things, the eyes move accordingly in order to have the object of interest portrayed on the forea. As a result of this, constantly moving the eyes from one screen to another might for example make a player miss some important information on one of the screens, which may lead to confusion during gameplay. Since AirConsole is a local multiplayer platform, the attention span of a player may be shortened due to the distractions from other players. Therefore, if the game developer fails to consider the imposed challenges, the overall impression from the gameplay might be ruined. This leads to the following problem formulation:

\begin{changemargin}{1cm}{1cm}
\textbf{How can the player’s attention be directed between a shared screen and a personal hand-held screen when there is a need to change their visual focus between the two in a local multiplayer game?}
\end{changemargin}

This problem formulation caused some additional questions to arise. These questions will be addressed throughout the project, and seek to ultimately answer the problem formulation:
\begin{itemize}
\item What effects do different types of sensory feedback have on the players' attention on a two-screen platform such as the AirConsole?
\item If given no indication of where to look, are players more likely to look at the phone screen while it is irrelevant to them?
\item If given no indication of where to look, are players more likely to fail to look at their phone screens while it is relevant to them?
\end{itemize}

The last two questions assume that in the final experiment of this project, test participants will play an AirConsole game which is designed in such a way that, when it is not a certain player’s turn, there is no relevant information for them on their phone screen. Thus, to get an overview of the state of the game, that player must aim their visual attention at the main screen.

Before an experiment addressing the problem formulation can be conducted, it is essential to have some knowledge about attention. Therefore, the next chapter will examine attention, the terms used to describe it, and methods which can be used to measure it.